% !TeX spellcheck = en_US
\chapter{集体流}
\section{cumulant method}
在重离子碰撞中观测粒子的集体运动是研究碰撞中产生的高温高密物质的有效方法。
集体运动的各向异性是非对心碰撞初期产生的物质在坐标空间的各向异性的结果,
动量空间的各向异性已经在RHIC,LHC实验中观测到。

实验上产生的粒子可以用表示为:
\begin{equation}
  \label{seveneq:1}
  E\frac{d^{3}N}{d^{3}p} = \frac{1}{2\pi}\frac{d^{2}N}{p_{t}dp_{t}dy}\left(1+\sum_{n=1}^{\infty}2v_{n}cos(n(\phi-\Psi_{R})) \right)
\end{equation}
其中$ E $是粒子的能量,
$ \vec{p} $为粒子的三动量,%
$ p_{t} $是粒子的横动量,%
$ \phi $是粒子的方位角,%
$ y $是粒子的快度,%
$ \Psi_{R} $事件的反应平面。%
公式\label{seveneq:1}中傅里叶系数$ v_{n} $表示各向异性流。%
一阶系数$ v_{1} $称为直接流。%
二阶系数$ v_{2} $称为椭圆流。%
另一方面如果我们只对平均值感兴趣的化,
我们可以将它们称为reference流。%
单个事件平均的傅里叶系数为:
$<v_{n}>$ 

%%% Local Variables:
%%% mode: latex
%%% TeX-master: "chapter07"
%%% End:
