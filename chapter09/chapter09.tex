 \chapter{零碎知识}

\section{碰撞的非弹种类}
According to hadronic cross section, elastic and in-elastic collisions are sorted.
For in-elastic collisions, there are singly-diffractive, double-diffractive and non-diffractive processes.
 Diffractive processes are difined as processes where one nucleon acts as a region of absorption and the interference of scattering amplitudes gives rise to diffraction pattern in the forward and backward regions. A nucleon suffering a diffractive scattering becomes excited and then loses a small amount of energy when breaking up into a few particles at a small emission angle. This can occur in one of the nucleons (singly) or in high both (doubly). In non-diffractive processes the nucleons hit "head-on" and both disintegrate creating large particle multiplicities at mid-rapidity. The STAR p$+$p trigger is only sensitive to the \textit{non-singly diffractive (NSD)} cross-section ($\sigma_{NSD}$), since it requires charged tracks to be detected in coincidence on both sides of the interaction point, so about 70\% of the inelastic cross-section ($\sigma_{inel}$) are measured at STAR.

 \section{V0 decays}
 The appearance of the decay of an unobserved neutral strange particle into two observed charged daughter particles gives rise to the terminology \textquotedblleft{}V0\textquotedblright{} to describe the decay topology.
The following neutral strange species have been analysed:\\
{\color{red}不可观测的中性奇异粒子衰变为两个带电荷的可观测的daughter particles, 我们用术语'V0'来描述这种衰变形式:}
\begin{table}[hbpt]
  \centering
  \begin{tabular}{ccc} \hline\hline
    Species & Decay channel & Branching ratio \\\hline
    $K^{0}_{S}$  & $\pi^{+} + \pi^{-}$ & 0.692 \\\hline
    $\Lambda$ & $p + \pi^{-}$ & 0.639 \\\hline
    $\bar{\Lambda}$ & $\bar{p} + \pi^{+}$ & 0.639 \\\hline
  \end{tabular}
  \caption{V0 decay}
  \label{tab:V0decay}
\end{table}
Candidate V0s are formed by combining together all possible pairs of opposite chapter-sign tracks in an event.
The invariant mass of the V0 candidate under different decay hypotheses can then be determined from the track momentum and the daughter masses(e.g. for $\Lambda$ the positive daughter is assumed to be a proton, the negative daughter a $\pi^{-}$).
The spectra contain three contributions:
\begin{itemize}
\item {\color{red}real particles of the species of interest;}
\item {\color{red}neutral strange particles of a different species;}
\item {\color{red}conbinatorial background from chance positive/negative track crossings.}
\end{itemize}
Selection cuts are the applied to the candidates to suppress the background whilst maintaining as much signal as possible.
There are two methods for reducing background.
\begin{itemize}
\item {\color{seagreen}energy-loss particle identification.}
\item {\color{seagreen}geometrical cuts on the V0 candidates.}
\end{itemize}

\section{衰变特性}
\begin{table}[hbpt]
  \centering
  \begin{tabular}{c|c|c}\hline
    重子 & 主要衰变方式 & 分值比(\%) \\\hline
    n & $p e^{-} \bar{\upsilon}_{e}$ & 100 \\\hline
    \multirow{2}*{$\Lambda$} & $p \pi^{-}$ & 63.9 \\\cline{2-3}
         & $n \pi^{0}$  & 35.8 \\\hline
    \multirow{2}*{$\Sigma^{+}$} & $p \pi^{0}$ & 51.57 \\\cline{2-3}
         & $n \pi^{+}$ & 48.31 \\\hline
    $\Sigma^{0}$ & $\Lambda \gamma$ & 100 \\\hline
    $\Sigma^{-}$ & $n \pi^{-}$ & 99.848 \\\hline
    $\Xi^{-}$ & $\Lambda \pi^{-}$ & 99.887 \\\hline
    $\Xi^{0}$ & $\Lambda \pi^{0}$ & 99.54 \\\hline
    \multirow{2}*{$\Omega$} & $\Lambda K^{-}$ & 67.8 \\\cline{2-3}
         & $\Xi \pi^{-}$ & 23.6 \\\hline
  \end{tabular}
  \caption{baryon}
  \label{tab:baryon}
\end{table}

\begin{table}[hbpt]
  \centering
  \begin{tabular}[htpb]{c|c|c}\hline

  \end{tabular}
  \caption{meson}
  \label{tab:meson}
\end{table}

%%% Local Variables:
%%% mode: latex
%%% TeX-master: "chapter09"
%%% End:
