\newcommand{\starScheduler}{\href{http://www.star.bnl.gov/public/comp/Grid/scheduler/manual.htm}{见链接}}
\chapter{STAR scheduler-User Manual}
\section{Introduction}
The STAR scheduler was developed to allow users to submit a job on several input files.
residing on NFS of on the various local disks of the farm machines.
The scheduler will divide the job into different processes that will be dispatched to different machines.
In order for the scheduler to do that,
the user has to follow some simple rules when writing his program,
and to write an XML file describing the job he wants to be done.

In this manual we show the steps the user needs to take to use the scheduler.

\section{An example}
\subsection{Preparing the program}
What will you basically submit to the scheduler is a command line.
The program will have to receive instruction from the scheduler which input files it should use.
For your convenience,
there are two ways this is done.
\subsubsection{Getting the input files from the filelist}
The scheduler will generate a file list like this for each process:
\begin{center}
\rowcolors{0}{seagreen}{seagreen}
\begin{tabular}{l}
/star/data21/reco/productionCentral/FullField/P02gc/2001/312/st\_physics\_2312011\_raw\_0017.MuDst.root\\
/star/data21/reco/productionCentral/FullField/P02gc/2001/312/st\_physics\_2312011\_raw\_0016.MuDst.root\\
/star/data21/reco/productionCentral/FullField/P02gc/2001/312/st\_physics\_2312011\_raw\_0015.MuDst.root
\end{tabular}
\end{center}

The environment variable \$FILELIST will be set to the name of the file list.
You can use this variable on the command line to retrieve the name as a parameter of your program.
For example, submitting cat \$FILELIST will display in the standard out all the input files that were assigned to that particular process.

Be careful that the output file is different for every process it might be dispatched,
otherwise two processes that were dispatched to different machines and are running in the same moment will interfere with each other.
You can, for example,
have the name of the output files depend on the filelist name,
or from the job ID,
which you can retrieve from the environment variable \$JOBID.
In case you have an output file for each input file,
and all your input files have different names,
you can have the output file depend on the particular input.
All the input files are assigned to only one process.

\subsubsection{Getting the input files from environment variables}
Another way to retrieve the file list is to look for the \$INPUTFILECOUNT variable that tells you how many files where assigned to the process.
Then you can look for the \$INPUTFILE0\dots\$INPUTFILEn for each input file name.
Also here you have to be careful for the output file names.

\subsection{Job description}
Once your program is ready, you can write the job description. A job description looks like this:
\begin{figure}[hbpt]
  \centering
  \includegraphics[width=\linewidth]{./pictures/pass.pdf}
  \caption{jobDescription}
  \label{fig:jobDescription}
\end{figure}
The first line tells the parser what set of characters was used.
Don't worry about it.
If you are not used to XML,
you can see that there are some tags that open and close.
The main tag you see is $<$job$>$ that closes at $<$/job$>$.
It means that everything inside that describes a job.
You could have more than one,
but you won't need to do that.
Not having specified the working directory,
the current directory will be used.

Inside the job, you can see another tag $<$command$>$.
Inside that you will put the command line you will want to be executed.
Then you have the $<$stdout$>$ tag:
it tells where the output of the process will be saved.
After that you see many $<$input$>$ tags.
These don't have a $<$/stdout$>$ or $<$/input$>$ because they are directly closed at the end (notice the \textquoteleft{/}\textquoteright in the ending \textquoteleft{/}$>$\textquoteright).
URL is an attribute of all the tags that will specify a file (input, stdout, stdin, stderr).
For example,
in the first input tag you are telling the scheduler that you require that file as an input for your job.
After \textquotedblleft{file:}\textquotedblright you have to put a full path to an NFS mounted file.

There are other options you can specify in your job description,
and you can find the details later in this manual.

\subsection{job submit}
Once you have prepared your XML file, you are ready. You just type:
\begin{center}
  \colorbox{seagreen}{star-submit jobDescription.xml}
\end{center}
where jobDescription.xml is the name of the file that contains your job description.
In this version,
you have to specify the position of the star-submit command.
In the final version, that won't be required.

The scheduler will create the scripts and the file lists for the processes.
These will be created in your directory.
Also this will change in the final version,
and they will be put in a scratch directory.

\section{Job description specification}
Here is the full specification of what the XML for the job description can contain.

\subsection{XML syntax}
You don't need to know XML very well to understand how to use the scheduler.
For the sake of being complete,
here are a few concept that you will see in this description.
It's not necessary to know them, but since XML is being more and more used,
it might satisfy a general curiosity.
Remember that HTML is kind of a subset of XML.

The basic notion in XML is the element (or entity),
which is basically two tags with the same name:
a start tag and an end tag.
For example, $<$job$>$\dots$<{/}$job$>$ is an element.

An element can have some attributes,
that better specify it.
For example, in $<$job title=\textquotedblleft My title\textquotedblright$>$ title is an attribute of the job element.
The attributes are specified in the start tag.

An element generally contain some data, that can be other elements,
or just some text, for example in:
\begin{center}
  \colorbox{seagreen}{$<$job$>$ <command>echo test$<$/command$>$ $<$/job$>$}
\end{center}
The element job contains the element command that in turn contain the data \textquotedblleft{echo test}\textquotedblright.
An element can also contain no data at all,
and just have some attributes.
For example, in $<$input URL=\textquotedblleft{}file:/star/u/carcassi/myfile\textquotedblright{}/$>$ we have the element input,
with a URL attribute, the ends right away.
That is the start tag and the end tag are put together in one tag (notice the \textquoteleft{}/\textquoteright{} in the end).

That is all the XML we are using in the scheduler.

\subsection{the $<$Job$>$ element}
Here is an example of the $<$job$>$ element with all the attributes:
\begin{center}
\rowcolors{0}{seagreen}{seagreen}
  \begin{tabular}{l}
    $<$job simulateSubmission =\textquotedblleft{}true\textquotedblright{} mail=\textquotedblleft{}true\textquotedblright{}$>$ \\
    \dots \\
    $<$/job$>$
  \end{tabular}
\end{center}
The $<$job$>$ element can be have the following attributes~\ref{attributeOfJob}:\\
\begin{sidewaystable}[h]
  \caption{The attributes of $<$job$>$ element}
  \begin{tabular}{|c|c|c|c|}\hline
    Attribute & Allowed values & Meaning & Default \\\hline
    datasetPlitting & \textquotedblleft{}eventBased\textquotedblright{} or \textquotedblleft{}fileBased\textquotedblright{} & \starScheduler & \textquotedblleft{}fileBased\textquotedblright{}\\\hline
    splitBy & Any Catalog key & \starScheduler & none \\\hline
    simulateSubmission & \textquotedblleft{}true\textquotedblright{} or \textquotedblleft{}false\textquotedblright{} & \starScheduler & false-by default the jobs are submitted \\\hline
    name & any string & \starScheduler & none \\\hline
    mail & \textquotedblleft{}true\textquotedblright{} or \textquotedblleft{}false\textquotedblright{} & \starScheduler & false-by default no mail is allowed \\\hline
    nProcesses & any integer & \starScheduler & Determined by the input \\\hline
    maxEvents & any integer & \starScheduler & infinite \\\hline
    minEvents & any integer & \starScheduler & N/A \\\hline
    softLimits & \textquotedblleft{}false\textquotedblright{} & \starScheduler & false \\\hline
    maxFilesPerProcess & any integer & \starScheduler & Infinite \\\hline
    minFilesPerProcess & any integer & \starScheduler & N/A \\\hline
    eventsPerHour & integer of floating point & \starScheduler & Infinite \\\hline
    filesPerHour & intgeger or floating point & \starScheduler & Infinite \\\hline
    filesListSyntax & \textquotedblleft{}paths\textquotedblright{} or \textquotedblleft{}rootd\textquotedblright{} or \textquotedblleft{}xrootd\textquotedblright{} & \starScheduler & paths \\\hline
    inputOrder & a string that is a catalog attribute & \starScheduler & none \\\hline
  \end{tabular}
  \label{attributeOfJob}
\end{sidewaystable}

Below is a list of dispatcher and the job tag attributes each supports.
It should be noted that the attributes (simulateSubmission, nProcesses, maxFilesPerProcess, minFilesPerProcess, filesPerHour, fileListSyntax, inputOrder) are supported by all dispatchers,
as they are observed at a higher level \starScheduler.

\subsection{The $<$command$>$ element}
The $<$command$>$ element doesn't have any attributes, and the data that it contains is the actual command script that will be submitted using a csh script.
%%% Local Variables:
%%% mode: latex
%%% TeX-master: "chapter01"
%%% End:
