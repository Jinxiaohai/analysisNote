\chapter{RHIC实验设备}
\section{加速器设备}
Colliding ions in RHIC is a multi-step process.
Negatively charged ions ($A^{-1}$ or $d^{-1}$ for example) from a pulsed sputter ion source are partially stripped of their eletrons and then accelerated in the Tandem van de Graaff.
After further stripping (for gold ions this corresponds to a charge state of $+$32) at the exit of the Tandem,
the ions are delivered to the Booster Synchrotron where they are accelerated more.
The ions are stripped again at the exit of the Booster (eg. gold ions reach a $+$77 charge state at this stage) and injected to the AGS for acceleration to the RHIC injection energy.
Fully stripped state ($+$79 for gold ions) is reached at the exit of the AGS.
In p$+$p collisions, the protons are injected into the Booster Synchrotron directly from the LINAC (LINear ACcelerator), accelerated in the AGS and finally injected in the RHIC.
The Collider itself consists of two concentric accelerator/storage rings on a horizontal plane, one for clockwise and the other for counter-clockwise beams.
The rings are oriented so that they intersect with one another at six locations along their 3.8 km circumference.
1740 superconducting magnets are required in order to bend,
focus and steer the beams to a co-linear path for head-on collisions.

\section{STAR Detector}
The Solenoidal Tracker at RHIC (STAR) was designed primarily for measurements of hadron production over a large solid angle,
featuring detector systems for high precision tracking,
momentum analysis,
and particle identification in a region surrounding the center-of-mass rapidity.
The large acceptance of STAR (complete azimuthal symmetry $\Delta\Phi = 2\pi$ and a pseudo-rapidity range |$\eta$| < 4.) makes it particularly well suited for single event characterization of heavy ion collisions and for the detection of hadron jets.
Its main components are a large Time Projection Chamber(TPC),
a Silicon Vertex Tracker (SVT),
two smaller radial Forward and Backward TPCs(FTPCs),
a Time of Flight patch (TOF) and an Electromagnetic Calorimeter (EMC) inside a 0.5T magnetic field.
%%% Local Variables:
%%% mode: latex
%%% TeX-master: "chapter08"
%%% End:
